\documentclass{beamer}

\mode<presentation>
\usepackage{beamerthemesplit}
\usepackage[latin1]{inputenc}
\usepackage{amsmath}
\usepackage{amsfonts}
\usepackage{amssymb}
\usepackage{makeidx}
\usepackage{graphicx}
\usepackage{float}
\usepackage{color}
\usepackage{multirow}
\usepackage{amsmath,amsthm,amssymb}
\usepackage{subfigure}
\usepackage[german]{babel}
\usepackage{xy}
\usetheme{antibes}
\title{Implimentation of Nine Men`s Morris}
\author{Lars Engel \newline Vikash\newline Ahsan Yousuf}

\institute{Fachhochschule Kiel}
\date{\today}

\begin{document}

\AtBeginSection[]
{
	\begin{frame}
	\footnotesize{
		\frametitle{Outline}
		\tableofcontents[currentsection]}
	\end{frame}
}

\begin{frame}
\titlepage
\end{frame}


\section{Introduction}
\begin{frame}{Introduction }
\begin{itemize}
\item KUKA lbr iiwa 7.
\item Game called Nine Men`s Morris.
\item Cognex Camera.
\item Artificial Intelligence.
\end{itemize}
\end{frame}

\section{Milestones of Project}
\begin{frame}
\begin{itemize}
\item Human vs KUKA robot.  
\item Robot can detect human moves and implement its moves wisely with AI.
\item Robot knows its turn after human.
\item through camera ``Cognex`` robot interects with real world.
\end{itemize}
\end{frame}

\section{Software and Hardware Tools}
\begin{frame}
\begin{itemize}
\item Sunrise Workbench by KUKA.
\item Eclipse for testing AI and server setup.
\item GIT for version control.
\item insight Explorer.
\item Robotic Arm by KUKA.
\item Interface to Cognex Camera.
\end{itemize}
\end{frame}

\section{Basic requirements to achieve target}
\begin{frame}
\begin{itemize}
\item Understanding of nine men`s Games Rules.
\item Get started with some useful methods of robot.
\item Learn and approach for insight explorer. 
\end{itemize}
\end{frame}


\section{Task distribution}
\begin{frame}
\begin{itemize}
\item Task 1: A simple program that move game piece from one point to another point (Sunrise Workbench).
\item Task 2: Image recognition with Insight explorer.
\item Task 3: Intersection points strategy.
\item Task 4: Game filed board measurement.
\item Task 5: code for game field board measurement.
\item Task 6: First program that pick game pieces from out of the game field and place in to the given point on to the game field.

\end{itemize}
\end{frame}

\begin{frame}{Task distribution Continue...}
\begin{itemize}
\item Task 7: program that wait until user play his moves than robot play its turn.
\item Task 8: Implementation of extra features for human safety.
\item Task 9: ideas for placing pieces in outside field (Horizontal or Vertical).
\item Task 10: Merger of our program with AI.
\item Task 11: Communication between robot and camera.
\end{itemize}
\end{frame}



\section{Implementation phase}
\subsection{Program Architecture}
\begin{frame}
``Class Diagram``
\begin{itemize}
\item 
\item 
\end{itemize}
\end{frame}



\section{Difficulties faced}
\begin{frame}
\begin{itemize}
\item understanding of robotics features of robot (Software).
\item Cartesian coordinates.
\item Safety Issues.
\item Understanding of AI.
\item piece recognition by insight explorer (still) due to light interference.
\item game board alignment.
\item communication of robot and camera.
\end{itemize}
\end{frame}


\section{Conclusion}
\begin{frame}
\begin{itemize}
\item Successfully done our target of basic game playing of Human Vs Robot.
\item Improvements required:\newline
1. detection of cheat of Human player.\newline
2. our game starts from one fixed point,future enhancement will be it start from any random point.\newline
3. for now human player always start the game, in future any one can start the game.\newline
4. Alignment of game board is fixed, in future rotation of field does not matter.
\end{itemize}
\end{frame}


\begin{frame}
\begin{center}  
\Huge \textbf{ThankYou}
\end{center}
\end{frame}

\end{document}

